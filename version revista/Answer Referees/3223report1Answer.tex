\documentclass[]{report}


\usepackage{amsmath} %General package for maths (loads also amsbsy to make bold symbols)
\usepackage{amsthm} %Package for theorems
\usepackage{amssymb} %Symbols (loads also amsfonts)


% Title Page
\title{}
\author{}


\begin{document}
	
	\begin{center}
		\textbf{	Report 2332 1}
	\end{center}
	Below you can find the answers to the comments in the referee report.
	
	\medskip
	(1) on pages 8-9 it is mentioned that the saddle solution provides an alternative counterexample to the conjecture of De Giorgi, due to a result by Jerison and Monneau. Minimizers in $\mathbb{R}^8$, hence counterexamples to the conjecturein $\mathbb{R}^9$, that are doubly invariant under rotations in $\mathbb{R}^4$ in the first four and last four coordinates are in fact already known in:
	
	Liu, Yong; Wang, Kelei; Wei, Juncheng. Global minimizers of the Allen--Cahn equation in dimension $n\geq 8$. \textit{J. Math. Pures Appl.} (9) \textbf{108} (2017),
	no. 6, 818--840.
	
	The saddle solution, however, when shown to a minimizer, provides a
	more symmetric counterexample.
	
	\bigskip
	
\textbf{
	We realized that this reference was missing shortly after submitting the paper. We have rewritten the paragraphs concerning this issue accordingly.
	}
	
	\bigskip
	
	(2) concerning the conjecture of De Giorgi in dimensions 4 and 5, the following result is known much earlier before Savin:
	
	Ghoussoub, Nassif; Gui, Changfeng. On De Giorgi’s conjecture in dimensions 4 and 5. \textit{Ann. of Math.  }(2) \textbf{157} (2003), no. 1, 313--334.
	
	It is proved that with the same limit assumption at infinity, anti-symmetric
	solutions are one-dimensional.
	\bigskip
	
\textbf{	
	We have added this reference.
	}
\end{document}          

