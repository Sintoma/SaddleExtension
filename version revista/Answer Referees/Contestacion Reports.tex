Referee report


Report 1

Comments:
(1) on pages 8-9 it is mentioned that the saddle solution provides an alternative counterexample to the conjecture of De Giorgi, due to a result by Jerison and Monneau. Minimizers in R 8
, hence counterexamples to the conjecturein R 9
, that are doubly invariant under rotations in R 4 in the first four and
last four coordinates are in fact already known in:
Liu, Yong; Wang, Kelei; Wei, Juncheng. Global minimizers of the AllenCahn equation in dimension n ≥ 8. J. Math. Pures Appl. (9) 108 (2017),
no. 6, 818–840.
The saddle solution, however, when shown to a minimizer, provides a
more symmetric counterexample.


We realized that this reference was missing short after submitting the paper. We have rewritten the paragraphs concerning this issue accordingly.


(2) concerning the conjecture of De Giorgi in dimensions 4 and 5, the following
result is known much earlier before Savin:
Ghoussoub, Nassif; Gui, Changfeng. On De Giorgi’s conjecture in dimensions 4 and 5. Ann. of Math. (2) 157 (2003), no. 1, 313–334.
It is proved that with the same limit assumption at infinity, anti-symmetric
solutions are one-dimensional.


We have added this reference.




Report 2


• In the Abstract, one should specify that µ ∈ (0, 1) and u : R2m → R.

We have done it

• Page 2, line 12 from the bottom, this part is a bit sketchy, perhaps one can elaborate more on this, clarifying why this is “probably” true, specifying what the open problem is, what the main conjectures are in comparison with the results already known, and so on.

We have modified the paragraph accordingly. We have devoted some lines to comment the known results concerning the classification of nonlocal minimal cones. This should clarify a little bit more why the Simons cone is expected to be a minimizer in dimensions $2m\geq8$. We have also stressed that there is no known example of nonsmooth minimizing nonlocal minimal cone, and that the Simons cone should be the main candidate.

• There is also a general approach to stability, monotonicity, and onedimensional symmetry in: O. Savin, E. Valdinoci, Some monotonicity results for minimizers in the calculus of variations. J. Funct. Anal. 264 (2013), no. 10, 2469–2496.

We have added this reference in the context of classification of stable nonlocal minimal cones.


• In Definition 2.1, perhaps one should recall that ∂0 was defined after (1.3).

We have added this reminder.


• After Definition 2.1, I think that one could add some examples of “extension narrow” pairs, as well as of pairs which do not satisfy this condition.

We have added two examples, as well as their corresponding illustrations, for the sake of clarity.


• I wonder if Remark 2.4 can also comprise the viscosity setting.

Propositions 1.4, 2.2, 2.3 are surely true for viscosity (super/sub)solutions, due to the simple form of the operator. However, since in great part of this paper we deal with smooth solutions we have preferred to simplify the setting as described in  Remark 2.4, which is enough for the purposes of the article. 

• End of page 25, “by a direct computation”.

Corrected

