\documentclass[]{report}


\usepackage{amsmath} %General package for maths (loads also amsbsy to make bold symbols)
\usepackage{amsthm} %Package for theorems
\usepackage{amssymb} %Symbols (loads also amsfonts)


% Title Page
\title{}
\author{}


\begin{document}
	
	\begin{center}
		\textbf{	Report 2332 2}
	\end{center}
	Below you can find the answers to the comments in the referee report.
	
	\medskip
$\bullet$  In the Abstract, one should specify that $\gamma \in (0, 1)$ and $u : \mathbb{R}^{2m} \to \mathbb{R}$.

\medskip
\textbf{We have added both things.}
\bigskip

$\bullet$  Page 2, line 12 from the bottom, this part is a bit sketchy, perhaps one can elaborate more on this, clarifying why this is ``probably'' true, specifying what the open problem is, what the main conjectures are in comparison with the results already known, and so on.

\medskip
\textbf{We have modified the paragraph accordingly. We have devoted some lines to comment some known results concerning the classification of nonlocal minimal cones. This should clarify a little bit more why the Simons cone is expected to be a minimizer in dimensions $2m\geq8$. We have also stressed that there is no known example of nonsmooth minimizing nonlocal minimal cone, and that the Simons cone should be the main candidate.}
\bigskip


$\bullet$  There is also a general approach to stability, monotonicity, and onedimensional symmetry in: O. Savin, E. Valdinoci, Some monotonicity results for minimizers in the calculus of variations. \textit{J. Funct. Anal.  }\textbf{264} (2013), no. 10, 2469–2496.

\medskip
\textbf{We have added this reference in the context of classification of the stable  nonlocal minimal cones.}
\bigskip





$\bullet$  In Definition 2.1, perhaps one should recall that ∂0 was defined after (1.3).

\medskip
\textbf{We have added this reminder.}
\bigskip




$\bullet$  After Definition 2.1, I think that one could add some examples of “extension narrow” pairs, as well as of pairs which do not satisfy this condition.

\medskip
\textbf{We have added two examples, as well as their corresponding illustrations, for the sake of clarity.}
\bigskip




$\bullet$  I wonder if Remark 2.4 can also comprise the viscosity setting.

\medskip
\textbf{ Propositions 1.4, 2.2, 2.3 are surely true for viscosity (super/sub) solutions, due to the simple form of the operator. However, since in great part of this paper we deal with smooth solutions we have preferred to simplify the setting as described in  Remark 2.4, which is enough for the purposes of the article. }
\bigskip


$\bullet$  End of page 25, ``by \underline{a} direct computation''.

\medskip
\textbf{Corrected.}
\bigskip
\end{document}          
